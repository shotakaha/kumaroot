\documentclass[article, paper=a0, jafontsize=30pt, twocolumn]{jlreq}

%% 和文フォントの設定
\usepackage[haranoaji]{luatexja-preset}

%% レイアウトの調整
\usepackage{geometry}
\geometry{
    left=50truemm,
    right=50truemm,
    top=50truemm,
    bottom=50truemm,
}

%% ヘッダー/フッターの装飾
\usepackage{fancyhdr}
\pagestyle{fancy}
\fancyhead[L]{テンプレート}
\fancyhead[C]{作成日:2023/06/11}
\fancyhead[R]{\includegraphics[height=5\zw]{example-image.pdf}}
\fancyfoot[L]{CC-BY-4.0}
\fancyfoot[C]{©️ KumaROOT / Shota Takahashi}
\fancyfoot[R]{\thepage}

\usepackage{mwe}
\usepackage{graphicx}
\usepackage{caption}

%% 本番では削除
\usepackage[japanese]{layout}
\usepackage{bxjalipsum}

%% 表紙の設定
\title{LuaLaTeX + jlreq}

% \renewcommand{\thefootnote}{\arabic{footnote}}
\makeatletter
% \let\@fnsymbol\@arabic
\renewcommand{\@fnsymbol}[1]{\@arabic{#1}}
\makeatother

\author{
    すごい著者 \thanks{すごい大学} \\ \and
    これまたすごい著者 \thanks{すごい研究所} \\ \and
    これまたすごい著者 \thanks{すごい研究所}
}

\begin{abstract}
LuaLaTeXとjlreqでA0ポスターを作ってみます。
jlreqのクラスオプションでページサイズをA0(\texttt{paper=a0})、
フォントサイズを30pt(\texttt{jafontsize=30pt})、
二段組(\texttt{twocolumn})にしています。
概要は\texttt{abstract}環境を使ってプリアンブルに書いています。
プリアンブルに書いておくと、\texttt{\\maketitle}したときに、二段組にならずに表示できます。
余白は\texttt{geometry}パッケージで設定しています。
最適な余白のサイズははっきりいってわかりません。
ただし、用紙の端まで目一杯に文字を詰めるのがよくないことは確かです。
\end{abstract}


\begin{document}

%% ヘッダー/フッターの設定
\thispagestyle{fancy}
\fancyfoot[L]{CC-BY-4.0}
\fancyfoot[C]{©️ KumaROOT / Shota Takahashi}
\fancyfoot[R]{\thepage}

%% タイトル/著者/概要は二段組しない
\maketitle

\section{背景}

研究の背景を簡潔に書いてください。
この研究の分野全体における位置付けと、課題が説明できるとよいです。

\jalipsum{iroha}
\jalipsum{iroha}
\jalipsum{iroha}

\section{見出し}

\blindtext
\jalipsum{iroha}
\jalipsum{iroha}

\section{見出し}

\jalipsum{iroha}
\jalipsum{iroha}
\jalipsum{iroha}

\begin{figure}
    \centering
    \includegraphics[width=0.9\linewidth]{example-image-16x9.pdf}
    \caption{mweを使ってサンプル図を挿入しています。図のサイズは、行幅の80\%から90\%にすると見栄えがよいです。}
\end{figure}

\section{見出し}

\jalipsum{iroha}
\jalipsum{iroha}
\jalipsum{iroha}
\jalipsum{iroha}

\section{見出し}

\jalipsum{iroha}
\jalipsum{iroha}
\jalipsum{iroha}
\jalipsum{iroha}

\begin{figure}
    \centering
    \includegraphics[width=0.9\linewidth]{example-image-16x9.pdf}
    \caption{mweを使ってサンプル図を挿入しています。図のサイズは、行幅の80\%から90\%にすると見栄えがよいです。}
\end{figure}


\section{まとめ}

\jalipsum{iroha}
\jalipsum{iroha}
\jalipsum{iroha}
\jalipsum{iroha}


\end{document}
