\documentclass[t,aspectratio=169]{beamer}
\usetheme{JuanLesPins}
%\usecolortheme{rose}
\useinnertheme{rectangles}

\usepackage[hybrid]{markdown}

% フォントの設定
\usepackage[no-math,deluxe,hiragino-pron]{luatexja-preset}
\setmonofont{MoralerspaceKrypton}
\setmonojfont{MoralerspaceKrypton}
\renewcommand{\kanjifamilydefault}{\gtdefault}
\usepackage{unicode-math}

% コードブロックの設定
\usepackage{xcolor}
\definecolor{gray}{rgb}{.9, .9, .9}

\usepackage{minted}
\usemintedstyle{material}
\setminted{
    fontfamily=tt,
    fontseries=upright,
    frame=leftline,
    linenos=true,
    bgcolor=gray,
}

% 表紙の情報
\title{スライドの作り方}
\author{qumasan}
\institute{KumaROOT}
\logo{ロゴ}
\date{\today}

\begin{document}

\begin{frame}
%\maketitle
\titlepage
\end{frame}

\section*{もくじ}
\begin{frame}
    \frametitle{もくじ}
    \tableofcontents
\end{frame}

\section{beamerの設定}
\begin{frame}[fragile]
    \frametitle{beamerの設定}
    \begin{minted}{latex}
\documentclass[t,aspectratio=169]{beamer}
\usetheme{Luebeck}
    \end{minted}
\end{frame}

\section{数式フォントの設定}
\begin{frame}[fragile]
    \frametitle{数式フォントの設定}
    \begin{minted}{latex}
\usepackage[no-math]{luatexja-fontspec}
\usepackage{unicode-math}
    \end{minted}
\end{frame}

\section{数式}
\begin{frame}
    \frametitle{数式}
    \begin{equation}
        E = mc^{2}
    \end{equation}

    \begin{align}
        E &= \sqrt{m^{2} + p^{2}}\\
        E^{2} &= m^{2} + p^{2}
    \end{align}

    \begin{equation}
    ∫dx = x
    \end{equation}

\end{frame}


\appendix

\section*{バックアップ}
\begin{frame}
    \frametitle{バックアップ}
    \tableofcontents
\end{frame}

% 使い方が確認できたら、フレームごとにファイルを作成する
% スライドの順番の入れ替えができるようになる
\section{日本語の設定}
\begin{frame}[fragile]
    \frametitle{日本語の設定}
    \begin{minted}{latex}
\usepackage[no-math,deluxe,hiragino-pron]{luatexja-preset}
\setmonofont{MoralerspaceKrypton}
\setmonojfont{MoralerspaceKrypton}
\renewcommand{\kanjifamilydefault}{\gtdefault}
    \end{minted}
\end{frame}


\section{minted}

文書内にコードサンプルを表示したいときの環境です。
コンパイルする際には\texttt{-shell-escape}オプションが必要です。
(\mintinline{shell}{$ latexmk -pvc -shell-escape})

\begin{listing}[H]
\begin{minted}[linenos=true,showspaces]{python}
from dataclasses import dataclass
from pathlib import Path

@dataclass
class Config:
    path

    def __post_init__(self):
        self.path = Path(self.path)

if __name__ == "__main__":
    c = Config("config.toml")

\end{minted}
\caption{Pythonのサンプル}
\end{listing}

\begin{listing}[H]
\begin{minted}{latex}
\documentclass{jlreq}
\usepackage{minted}
\begin{document}

\end{document}
\end{minted}
\caption{LaTeXのサンプル}
\end{listing}

\section{表紙を作成する(maketitle)}

\begin{minted}{latex}
\documentclass{jlreq}

\title{LuaLaTeX + jlreq}
\author{Quma}
\date{\today}

\begin{document}

\maketitle

\end{document}
\end{minted}

\section{箇条書き(itemize)}

\begin{minted}{latex}
\begin{itemize}
    \item レベル1 - 箇条書き1
    \begin{itemize}
        \item レベル2 - 箇条書き1
        \begin{itemize}
            \item レベル3 - 箇条書き1
            \begin{itemize}
                \item レベル4 - 箇条書き1
                \item レベル4 - 箇条書き2
                \item レベル4 - 箇条書き3
            \end{itemize}
            \item レベル3 - 箇条書き2
            \item レベル3 - 箇条書き3
        \end{itemize}
        \item レベル2 - 箇条書き2
        \item レベル2 - 箇条書き3
    \end{itemize}
    \item[★] レベル1 - 箇条書き2
    \item レベル1 - 箇条書き3
\end{itemize}
\end{minted}

\begin{itemize}
  \item レベル1 - 箇条書き1
    \begin{itemize}
      \item レベル2 - 箇条書き1
        \begin{itemize}
          \item レベル3 - 箇条書き1
            \begin{itemize}
              \item レベル4 - 箇条書き1
              \item レベル4 - 箇条書き2
              \item レベル4 - 箇条書き3
            \end{itemize}
          \item レベル3 - 箇条書き2
          \item レベル3 - 箇条書き3
        \end{itemize}
      \item レベル2 - 箇条書き2
      \item レベル2 - 箇条書き3
    \end{itemize}
  \item レベル1 - 箇条書き2
  \item レベル1 - 箇条書き3
\end{itemize}

\section{箇条書き(enumerate)}
\begin{frame}[fragile]
    \frametitle{箇条書き(enumerate)}
    \begin{columns}
        \column{0.45\textwidth}
        \begin{enumerate}
            \item ばなな
            \item りんご
            \item ぶどう
        \end{enumerate}
        \column{0.45\textwidth}
        \begin{minted}{latex}
\begin{enumerate}
    \item ばなな
    \item りんご
    \item ぶどう
\end{enumerate}
        \end{minted}
    \end{columns}
\end{frame}

\section{箇条書き(description)}

\begin{minted}{latex}
\begin{description}
    \item[ラベル1] レベル1 - 箇条書き1
    \begin{description}
        \item[ラベル2] レベル2 - 箇条書き1
        \begin{description}
            \item[ラベル3] レベル3 - 箇条書き1
            \begin{description}
                \item[ラベル4] レベル4 - 箇条書き1
                \item[ラベル5] レベル4 - 箇条書き2
                \item[ラベル6] レベル4 - 箇条書き3
            \end{description}
            \item[ラベル7] レベル3 - 箇条書き2
            \item[ラベル8] レベル3 - 箇条書き3
        \end{description}
        \item[ラベル9] レベル2 - 箇条書き2
        \item[ラベル10] レベル2 - 箇条書き3
    \end{description}
    \item[ラベル11] レベル1 - 箇条書き2
    \item[ラベル12] レベル1 - 箇条書き3
\end{description}
\end{minted}

\begin{description}
  \item[ラベル1] レベル1 - 箇条書き1
    \begin{description}
      \item[ラベル2] レベル2 - 箇条書き1
        \begin{description}
          \item[ラベル3] レベル3 - 箇条書き1
            \begin{description}
              \item[ラベル4] レベル4 - 箇条書き1
              \item[ラベル5] レベル4 - 箇条書き2
              \item[ラベル6] レベル4 - 箇条書き3
            \end{description}
          \item[ラベル7] レベル3 - 箇条書き2
          \item[ラベル8] レベル3 - 箇条書き3
        \end{description}
      \item[ラベル9] レベル2 - 箇条書き2
      \item[ラベル10] レベル2 - 箇条書き3
    \end{description}
  \item[ラベル11] レベル1 - 箇条書き2
  \item[ラベル12] レベル1 - 箇条書き3
\end{description}






\end{document}
