\documentclass[t,aspectratio=169]{beamer}
\usetheme{Luebeck}

%\usepackage{luatexja}
\usepackage[no-math,deluxe,hiragino-pron]{luatexja-preset}
\renewcommand{\kanjifamilydefault}{\gtdefault}
\setmonofont{MoralerspaceKrypton}
\setmonojfont{MoralerspaceKrypton}

\usepackage{xcolor}
\definecolor{gray}{rgb}{.9, .9, .9}

\usepackage{minted}
\usemintedstyle{material}
\setminted{
    fontfamily=tt,
    fontseries=upright,
    frame=leftline,
    linenos=true,
    bgcolor=gray,
}

\title{スライドの作り方}
\author{qumasan}
\institute{qumasan}
\date{\today}


\begin{document}

\begin{frame}
\maketitle
\end{frame}

\section*{もくじ}
\begin{frame}
    \frametitle{もくじ}
    \tableofcontents
\end{frame}

\section{beamerの設定}
\begin{frame}[fragile]
    \frametitle{beamerの設定}
    \begin{minted}{latex}
\documentclass[t,aspectratio=169]{beamer}
\usetheme{Luebeck}
    \end{minted}
\end{frame}

\section{日本語の設定}
\begin{frame}[fragile]
    \frametitle{日本語の設定}
    \begin{minted}{latex}
\usepackage[no-math,deluxe,hiragino-pron]{luatexja-preset}
\setmonofont{MoralerspaceKrypton}
\setmonojfont{MoralerspaceKrypton}
\renewcommand{\kanjifamilydefault}{\gtdefault}
    \end{minted}
\end{frame}

\section{表紙の設定}
\begin{frame}[fragile]
    \frametitle{表紙の設定}
    \begin{columns}
        \column{0.45\textwidth}
        \begin{minted}{latex}
% プリアンブル
\title{タイトル}
\author{名前}
\institute{所属}
\date{\today}
        \end{minted}

        \column{0.45\textwidth}
        \begin{minted}{latex}
% 本文
\maketitle
        \end{minted}
    \end{columns}
\end{frame}

\section{コードブロックの設定}
\begin{frame}[fragile]
    \frametitle{コードブロックの設定}
    \begin{minted}{latex}
\usepackage{minted}
\setmintedstyle{material}
    \end{minted}
\end{frame}

\section{箇条書き(itemize)}
\begin{frame}[fragile]
    \frametitle{箇条書き(itemize)}
    \begin{columns}
        \column{0.45\textwidth}
        \begin{itemize}
            \item ばなな
            \item りんご
            \item ぶどう
        \end{itemize}
        \column{0.45\textwidth}
        \begin{minted}{latex}
\begin{itemize}
    \item ばなな
    \item りんご
    \item ぶどう
\end{itemize}
        \end{minted}
    \end{columns}
\end{frame}

\section{箇条書き(enumerate)}
\begin{frame}[fragile]
    \frametitle{箇条書き(enumerate)}
    \begin{columns}
        \column{0.45\textwidth}
        \begin{enumerate}
            \item ばなな
            \item りんご
            \item ぶどう
        \end{enumerate}
        \column{0.45\textwidth}
        \begin{minted}{latex}
\begin{enumerate}
    \item ばなな
    \item りんご
    \item ぶどう
\end{enumerate}
        \end{minted}
    \end{columns}
\end{frame}

\section{箇条書き(description)}
\begin{frame}[fragile]
    \frametitle{箇条書き(description)}
    \begin{columns}
        \column{0.45\textwidth}
        \begin{description}
            \item [ばなな] バショウ科
            \item [りんご] バラ科
            \item [ぶどう] ブドウ科
        \end{description}
        \column{0.45\textwidth}
        \begin{minted}{latex}
\begin{description}
    \item [ばなな] バショウ科
    \item [りんご] バラ科
    \item [ぶどう] ブドウ科
\end{description}
        \end{minted}
    \end{columns}
\end{frame}

\end{document}
