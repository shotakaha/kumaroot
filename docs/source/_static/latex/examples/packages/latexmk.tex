\section{タイプセットの設定(latexmkrc)}

\texttt{latexmk}というLaTeX版makeコマンドがあります。
タイプセットに必要な設定を\texttt{latexmkrc}に記述すると、LaTeXエンジンなどに関係なく常に\texttt{latexmk}でタイプセット(コンパイル)できるようになります。

\begin{minted}{bash}
$pdf_mode = 4;
$preview_continuous_mode = 1;
$pvc_timeout = 1;
$pvc_timeout_mins = 10;  # 30min; default
$sleep_time = 60;  # 60s
# $out_dir = "outd";
# $aux_dir = "auxd";
\end{minted}

\subsection{エンジンを設定する(\$pdf\_mode)}

\texttt{LuaLaTeX}を使う場合は、\texttt{\$pdf\_mode = 4;}と記述します。

\subsection{コンパイルするファイルを指定したい(@default\_files)}

\texttt{@default\_files}にファイルを指定できます。
拡張子(\texttt{.tex})はつけても、つけなくてもよいです。
ファイル名を指定しない場合は、ディレクトリ内にあるTeXファイルが対象となります。

\subsection{ライブプレビューしたい(\$preview\_continuous\_mode)}

ライブプレビューをするには、\texttt{\$preview\_continuous\_mode = 1}とします。
コンパイル対象になっているファイルに変更を加えたときに、自動でタイプセットしてくれます。
コマンドラインの\texttt{-pvc (preview continuously)}オプションでも指定できます。

\subsection{シェルエスケープしたい(``\-shell\-escape``)}

タイプセット時に外部コマンドを呼び出す場合は\texttt{\-shell\-escape}オプションが必要です。
セキュリティの観点からこのオプションは設定ファイルで常時有効にするのではなく、コマンド実行時にその都度有効にするのがよいと思います。

\begin{minted}{bash}
$ latexmk -shell-escape
\end{minted}
