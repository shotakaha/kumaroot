\documentclass{jlreq}

\usepackage{graphicx}

\usepackage{geometry}
\geometry{headheight=3cm}

\usepackage{fancyhdr}
\pagestyle{fancy}
\fancyhead[R]{\includegraphics[height=2cm]{../../../quma.jpeg}}
\fancyfoot[C]{©️ KumaROOT / Shota Takahashi}
\fancyfoot[R]{\thepage}


\usepackage{bxjalipsum}

\title{LuaLaTeX + jlreq}
\author{すごい著者 \thanks{所属とメールアドレス}}

\begin{document}

\begin{titlepage}
    \centering
    \includegraphics[width=0.6\textwidth]{../../../quma.jpeg} % ロゴを挿入する場合
    \vspace{2cm} % 適宜、上下のスペースを調整
    {\Huge\bfseries LuaLaTeX\par} % 大きなタイトル
    \vspace{1.5cm} % 適宜、上下のスペースを調整
    {\Large すごい著者\par} % 著者名
    \vfill % 縦方向に余白を均等に分割
    {\large \today\par} % 日付を表示する場合
\end{titlepage}

\maketitle
\tableofcontents

\begin{abstract}
エンジンにLuaLaTeX、ドキュメントクラスにjlreqを使ったサンプルです。
CTANで公開されているパッケージのみを使用し、独自パッケージなどは使わないつもりです。
このソースをそのままLaTeXファイルに保存してコンパイルすればPDFができるはずです。

\end{abstract}

\section{寿限無}

\jalipsum{jugemu}

\section{草枕}

\jalipsum{kusamakura}

\end{document}
