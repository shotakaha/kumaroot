\documentclass{jlreq}

\usepackage{graphicx}

\usepackage{geometry}
\geometry{
    headheight=6\zh,
    headsep=5\zh,
    includehead=true,
}

\usepackage{fancyhdr}
\pagestyle{fancy}
\fancyhead[L]{テンプレート}
\fancyhead[C]{作成日:2023/06/11}
\fancyhead[R]{\includegraphics[height=5\zh]{../../../quma.jpeg}}
\fancyfoot[C]{©️ KumaROOT / Shota Takahashi}
\fancyfoot[R]{\thepage}

\usepackage[japanese]{layout}
\usepackage{bxjalipsum}

\title{LuaLaTeX + jlreq}
\author{すごい著者 \thanks{所属とメールアドレス}}

\begin{document}

\maketitle

\begin{abstract}
エンジンにLuaLaTeX、ドキュメントクラスにjlreqを使ったサンプルです。
CTANで公開されているパッケージのみを使用し、独自パッケージなどは使わないつもりです。
このソースをそのままLaTeXファイルに保存してコンパイルすればPDFができるはずです。

\end{abstract}

\section{寿限無}

\jalipsum{jugemu}

\section{草枕}

\jalipsum{kusamakura}

\newpage

\layout

\end{document}
