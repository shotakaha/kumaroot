\documentclass[article, head_space=25truemm, foot_space=15truemm, gutter=15truemm]{jlreq}

%% 和文フォントの設定
\usepackage[haranoaji,deluxe]{luatexja-preset}
\setmonofont{MoralerSpaceKrypton}
\setmonojfont{MoralerSpaceKrypton}

%% ページ設定
\usepackage[showframe]{geometry}
\geometry{
    headheight=5\zw,
}
\usepackage[
    definitionLists=true,
]{markdown}

%% ヘッダー/フッターの装飾
\usepackage{fancyhdr}
\pagestyle{fancy}
\fancyhead[L]{テンプレート}
\fancyhead[C]{作成日:2023/06/11}
\fancyhead[R]{\includegraphics[height=5\zw]{../../../quma.jpeg}}
\fancyfoot[L]{CC-BY-4.0}
\fancyfoot[C]{©️ KumaROOT / Shota Takahashi}
\fancyfoot[R]{\thepage}

%% 見出しの設定
%\usepackage{titlesec}
%\titleformat{\section}{}{\Large\bfseries}{\thesection}{1em}

% 箇条書き
\usepackage{enumitem}
\setlist[itemize]{
  labelindent=1em,
  leftmargin=*,
  rightmargin=2em,
  labelsep=1em,
}
\setlist[itemize, 1]{
  label=$\triangleright$,
}
\setlist[enumerate]{
  labelindent=1em,
  leftmargin=*,
  rightmargin=2em,
  labelsep=1em,
}
\setlist[description]{
  labelindent=2em,
  itemindent=-2em,
  leftmargin=*,
  rightmargin=2em,
  labelsep=1em,
  font=\ttfamily\bfseries,
}



%% 参考文献
\usepackage{biblatex}
\addbibresource{参考文献.bib}

%% 図版の設定
\usepackage{graphicx}
\usepackage[dvipsnames]{xcolor}
\usepackage{minted}
\setminted{
    fontfamily=tt,
    fontseries=upright,
    frame=leftline,
    linenos=true,
    bgcolor=black!10,
    showspaces=true,
    spacecolor=black!50,
    stripall=true,
    stripnl=true,
}


%% 本番では削除
\usepackage[japanese]{layout}
\usepackage{bxjalipsum}

\usepackage{hyperref}

%% 表紙の設定
\title{LuaLaTeX + jlreq}

\author{
    すごい著者 \thanks{すごい大学} \\ \and
    これまたすごい著者 \thanks{こっちの研究所} \\ \and
    またまたすごい著者 \thanks{あっちの研究所}
}
\date{\today}


% 本文ここから
\begin{document}

% 表紙を出力
\maketitle

% 概要
\begin{abstract}
エンジンにLuaLaTeX、ドキュメントクラスにjlreqを使ったサンプルです。
CTANで公開されているパッケージのみを使用し、独自パッケージなどは使わないつもりです。
このソースをそのままLaTeXファイルに保存してコンパイルすればPDFができるはずです。
\end{abstract}

\tableofcontents

\section{ページ設定}

\subsection{geometry}

\subsection{markdown}

\begin{markdown}
```latex
\usepackage[
    definitionLists=true,
    ]{markdown}
```
\end{markdown}

\begin{markdown}
`definitionLists=true`
: `description`環境を有効にした
\end{markdown}

\section{見出しの設定}

\begin{minted}{latex}
    \usepackage{titlesec}
\end{minted}

\jalipsum{jugemu}

\subsection{項}

\jalipsum{iroha}

\subsubsection{目}

\jalipsum{jugemu}

\paragraph{段落}

\jalipsum{iroha}

\subparagraph{小段落}

\jalipsum{jugemu}





\section{箇条書き}

\subsection{enumitemの設定}

\inputminted{latex}{./preamble/enumitem.tex}

\begin{markdown}
1. ラベル(番号や用語)と説明文の間隔を1文字分に変更した
\end{markdown}

\subsection{出力例}

\subsubsection{itemize}

\begin{minted}{latex}
\begin{itemize}
    \item \jalipsum{iroha}
    \item \jalipsum{jugemu}
    \item \jalipsum{iroha}
    \item \jalipsum{jugemu}
\end{itemize}
\end{minted}

\begin{itemize}
    \item \jalipsum{iroha}
    \item \jalipsum{jugemu}
    \item \jalipsum{iroha}
    \item \jalipsum{jugemu}
\end{itemize}

\paragraph{markdown環境}

\begin{markdown}
```latex
- \jalipsum{iroha}
- \jalipsum{jugemu}
- \jalipsum{iroha}
- \jalipsum{jugemu}
```
\end{markdown}

\begin{markdown}
- \jalipsum{iroha}
- \jalipsum{jugemu}
- \jalipsum{iroha}
- \jalipsum{jugemu}
\end{markdown}

\subsubsection{enumerate}

\begin{minted}{latex}
\begin{enumerate}
    \item \jalipsum{iroha}
    \item \jalipsum{jugemu}
    \item \jalipsum{iroha}
    \item \jalipsum{jugemu}
\end{enumerate}
\end{minted}

\begin{enumerate}
    \item \jalipsum{iroha}
    \item \jalipsum{jugemu}
    \item \jalipsum{iroha}
    \item \jalipsum{jugemu}
\end{enumerate}

\paragraph{markdown環境}

\begin{markdown}
```latex
1. \jalipsum{iroha}
1. \jalipsum{jugemu}
1. \jalipsum{iroha}
1. \jalipsum{jugemu}
```
\end{markdown}

\begin{markdown}
1. \jalipsum{iroha}
1. \jalipsum{jugemu}
1. \jalipsum{iroha}
1. \jalipsum{jugemu}
\end{markdown}


\subsubsection{description}

\begin{minted}{latex}
\begin{description}
    \item[いろは唄] \jalipsum{iroha}
    \item[寿限無] \jalipsum{jugemu}
    \item[いろは唄] \jalipsum{iroha}
    \item[寿限無] \jalipsum{jugemu}
\end{description}
\end{minted}

\begin{description}
    \item[いろは唄] \jalipsum{iroha}
    \item[寿限無] \jalipsum{jugemu}
    \item[いろは唄] \jalipsum{iroha}
    \item[寿限無] \jalipsum{jugemu}
\end{description}

\paragraph{markdown環境}

\begin{markdown}
```latex
いろは唄
: \jalipsum{iroha}

寿限無
: \jalipsum{jugemu}

いろは唄
: \jalipsum{iroha}

寿限無
: \jalipsum{jugemu}
```
\end{markdown}

\begin{markdown}
いろは唄
: \jalipsum{iroha}

寿限無
: \jalipsum{jugemu}

いろは唄
: \jalipsum{iroha}

寿限無
: \jalipsum{jugemu}
\end{markdown}

\section{コードブロックの設定}

\begin{minted}{latex}
\usepackage{minted}
\setminted{
    fontfamily=tt,
    fontseries=upright,
    frame=leftline,
    linenos=true,
    bgcolor=black!10,
    showspaces=true,
    spacecolor=black!50,
    stripall=true,
    stripnl=true,
}
\end{minted}

\begin{markdown}
`fontfamily=tt':
コード内のフォント(とくにコメント)をモノスペース体/立体に変更した

- コードブロックの左側に線を追加した
- 行番号を表示した
\end{markdown}

\section{例文の設定}

\begin{minted}{latex}
    \usepackage{bxjalipsum}
\end{minted}

\subsection{いろは唄}

\begin{minted}{latex}
    \jalipsum{iroha}
\end{minted}
\jalipsum{iroha}

\subsection{寿限無}

\begin{minted}{latex}
    \jalipsum{jugemu}
\end{minted}
\jalipsum{jugemu}

\subsection{草枕}

\begin{minted}{latex}
    \jalipsum{kusamakura}
\end{minted}

\jalipsum{kusamakura}

\subsection{吾輩は猫である}

\begin{minted}{latex}
    \jalipsum{wagahai}
\end{minted}

\jalipsum{wagahai}

%% レイアウトを確認
\section{レイアウト}

\begin{minted}{latex}
    \usepackage[japanese]{layout}
\end{minted}

\layout

%% 参考文献を出力
\printbibliography[title={参考文献}]

\end{document}
