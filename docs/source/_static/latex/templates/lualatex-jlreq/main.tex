\documentclass[
    article,
    head_space=25truemm,
    foot_space=10truemm,
    gutter=15truemm]{jlreq}

%% 和文フォントの設定
\usepackage[haranoaji,deluxe]{luatexja-preset}
\setmonofont{MoralerSpaceKrypton}
\setmonojfont{MoralerSpaceKrypton}

%% ページ設定
\usepackage[showframe]{geometry}
\geometry{
    headheight=6\zw,
}

%% ヘッダー/フッターの装飾
\usepackage{fancyhdr}
\pagestyle{fancy}
\fancyhead[L]{テンプレート}
\fancyhead[C]{作成日:2023/06/11}
\fancyhead[R]{\includegraphics[height=5\zw]{../../../quma.jpeg}}
\fancyfoot[L]{CC-BY-4.0}
\fancyfoot[C]{©️ KumaROOT / Shota Takahashi}
\fancyfoot[R]{\thepage}

%% 見出しの設定
%\usepackage{titlesec}
%\titleformat{\section}{}{\Large\bfseries}{\thesection}{1em}

% 箇条書き
\usepackage{enumitem}
\setlist[itemize]{
  labelindent=1em,
  leftmargin=*,
  rightmargin=2em,
  labelsep=1em,
}
\setlist[itemize, 1]{
  label=$\triangleright$,
}
\setlist[enumerate]{
  labelindent=1em,
  leftmargin=*,
  rightmargin=2em,
  labelsep=1em,
}
\setlist[description]{
  labelindent=2em,
  itemindent=-2em,
  leftmargin=*,
  rightmargin=2em,
  labelsep=1em,
  font=\ttfamily\bfseries,
}

% markdownパッケージの設定
% enumitemの設定より後に読み込む
\usepackage[
    definitionLists=true,
    gfmAutoIdentifiers=true,
    strikeThrough=true,
    stripIndent=true,
    subscripts=true,
    superscripts=true,
    taskLists=true,
    texMathDollars=true,
]{markdown}


%% 図版の設定
\usepackage{graphicx}
\usepackage[dvipsnames]{xcolor}

\section{mintedの設定}

\inputminted{latex}{preamble/minted.tex}

\begin{markdown}
  `fontfamily=tt` / `fontseries=upright`
  : コード内のフォント(とくにコメント)をモノスペース体/立体に変更した

  `frame=leftline`
  : コードブロックの左端に実践を追加した

  `linenos=true`
  : 行番号を表示した

  `bgcolor=black!10`
  : 背景色を設定した(黒10パーセント)

  `bgcolorpadding=1em`
  : 背景のパディング(内側の余白)を1文字分に設定した
\end{markdown}

文書にコードブロックを表示する時に便利なパッケージです。
\texttt{minted}環境や\texttt{mint}コマンド、\texttt{mintinline}コマンドが使えるようになります。
また\texttt{listing}環境も改良されます。

コンパイルする際には\texttt{-shell-escape}オプションが必要です。
(\mintinline{shell}{$ latexmk -pvc -shell-escape})

\begin{listing}[H]
\begin{minted}[linenos=true,showspaces]{python}
from dataclasses import dataclass
from pathlib import Path

@dataclass
class Config:
    path

    def __post_init__(self):
        self.path = Path(self.path)

if __name__ == "__main__":
    c = Config("config.toml")

\end{minted}
  \caption{Pythonのサンプル}
\end{listing}

\begin{listing}[H]
\begin{minted}{latex}
\documentclass{jlreq}
\usepackage{minted}
\begin{document}

\end{document}
\end{minted}
  \caption{LaTeXのサンプル}
\end{listing}


%% 参考文献
\usepackage{biblatex}
\addbibresource{参考文献.bib}

\usepackage[italic]{hepnames}

%% 本番では削除
\usepackage[japanese]{layout}
\usepackage{bxjalipsum}

\usepackage{hyperref}
\hypersetup{
    unicode=true
}

%% 表紙の設定
\title{LuaLaTeX + jlreq}

\author{
    すごい著者 \thanks{すごい大学} \\ \and
    これまたすごい著者 \thanks{こっちの研究所} \\ \and
    またまたすごい著者 \thanks{あっちの研究所}
}
\date{\today}


% 本文ここから
\begin{document}

% 表紙を出力
\maketitle

% 概要
\begin{abstract}
エンジンにLuaLaTeX、ドキュメントクラスにjlreqを使ったサンプルです。
CTANで公開されているパッケージのみを使用し、独自パッケージなどは使わないつもりです。
このソースをそのままLaTeXファイルに保存してコンパイルすればPDFができるはずです。
\end{abstract}

\tableofcontents

\section{ドキュメントクラスの設定}

このファイルは examples/main.tex に移行することにしました





\section{ページ設定}

\subsection{フォントの設定}

\subsection{geometryの設定}

\subsection{markdownの設定}

\inputminted{latex}{preamble/markdown.tex}

\begin{markdown}
`definitionLists=true`
: `description`環境を有効にした

`gfmAutoIdentifiers=true`
: GitHub Flavored Markdown の自動検出を有効にした

`strikeThrough=true`
: ~~打ち消し線~~を有効にした
\end{markdown}

\section{例文の設定}

\begin{minted}{latex}
    \usepackage{bxjalipsum}
\end{minted}

\subsection{いろは唄}

\begin{minted}{latex}
    \jalipsum{iroha}
\end{minted}
\jalipsum{iroha}

\subsection{寿限無}

\begin{minted}{latex}
    \jalipsum{jugemu}
\end{minted}
\jalipsum{jugemu}

\subsection{初恋}

\begin{minted}{latex}
    \jalipsum{hatsukoi}
\end{minted}
\jalipsum{hatsukoi}


%% レイアウトを確認
\section{レイアウト}

\begin{minted}{latex}
    \usepackage[japanese]{layout}
\end{minted}

\centering
\layout

%% 参考文献を出力
\printbibliography[title={参考文献}]

\end{document}
