\documentclass[article, line_length=40zw, head_space=30mm, foot_space=30mm, gutter=30mm]{jlreq}

%% 和文フォントの設定
\usepackage[haranoaji]{luatexja-preset}

\usepackage{graphicx}

%% レイアウトの調整
\usepackage{geometry}
\geometry{
    headheight=5\zw,
}

%% ヘッダー/フッターの装飾
\usepackage{fancyhdr}
\pagestyle{fancy}
\fancyhead[L]{テンプレート}
\fancyhead[C]{作成日:2023/06/11}
\fancyhead[R]{\includegraphics[height=5\zw]{../../../quma.jpeg}}
\fancyfoot[L]{CC-BY-4.0}
\fancyfoot[C]{©️ KumaROOT / Shota Takahashi}
\fancyfoot[R]{\thepage}

%% 本番では削除
\usepackage[japanese]{layout}
\usepackage{bxjalipsum}

%% 表紙の設定
\title{LuaLaTeX + jlreq}
\author{すごい著者 \thanks{所属とメールアドレス} \and すごい著者 \thanks{所属とメールアドレス}}


\begin{document}

%% 表紙を出力
\maketitle

\begin{abstract}
エンジンにLuaLaTeX、ドキュメントクラスにjlreqを使ったサンプルです。
CTANで公開されているパッケージのみを使用し、独自パッケージなどは使わないつもりです。
このソースをそのままLaTeXファイルに保存してコンパイルすればPDFができるはずです。
\end{abstract}

\section{寿限無}

\jalipsum{jugemu}

\section{草枕}

\jalipsum{kusamakura}

%% レイアウトを確認
\newpage
\layout

\end{document}
