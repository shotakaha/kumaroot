\documentclass[report]{jlreq}

\usepackage[Bjornstrup]{fncychap}
\usepackage{physics}

\usepackage{hyperref}
%% 目次のリンク色を変更する
\hypersetup{colorlinks=true}
\hypersetup{allcolors=blue}
%% PDFのしおりに章番号を追加する
\hypersetup{bookmarksnumbered=true}
%% PDFのメタ情報を設定する(オプション)
\hypersetup{pdftitle=修論のタイトル}
\hypersetup{pdfauthor=著者}
\hypersetup{pdfsubject=サブジェクト}
%\hypersetup{pdfcreator=LaTeX with hyperref}
%\hypersetup(pdfkeywords="キーワード1,キーワード2,...")
%\hypersetup(pdfduplex="Simplex|DuplexFlipShortEdge|DuplexFlipLongEdge"")  % 両面印刷の設

\usepackage{bxjalipsum}

\title{修士論文のテンプレート}
\author{自分の名前}

\begin{document}

\maketitle

\tableofcontents

%%%%%%%%%%%%%%%%%%%%%%%%%%%%%%%%%%%%%%%%%%%%%%%%%%
\chapter{いろは唄}

\jalipsum{iroha}

%%%%%%%%%%%%%%%%%%%%%%%%%%%%%%%%%%%%%%%%%%%%%%%%%%
\chapter{じゅげむ}

\jalipsum{jugemu}

%%%%%%%%%%%%%%%%%%%%%%%%%%%%%%%%%%%%%%%%%%%%%%%%%%
\chapter{憲法}

\jalipsum{preamble}

%%%%%%%%%%%%%%%%%%%%%%%%%%%%%%%%%%%%%%%%%%%%%%%%%%
\chapter{夏目漱石}

\section{吾輩は猫である}
\jalipsum{wagahai}

\section{草枕}
\jalipsum{kusamakura}

%%%%%%%%%%%%%%%%%%%%%%%%%%%%%%%%%%%%%%%%%%%%%%%%%%
\chapter{島崎藤村}

\jalipsum{hatsukoi}

%%%%%%%%%%%%%%%%%%%%%%%%%%%%%%%%%%%%%%%%%%%%%%%%%%
\chapter{ニュートリノ振動}

ニュートリノ振動は、フレーバーの固有状態($\nu_{e}, \nu_{\mu}, \nu_{\tau}$)と
質量の固有状態($\nu_{1}, \nu_{2}, \nu_{3}$)が一致せず、
さらに3つの質量固有状態が縮退していない場合に起こる。
この場合、混合状態は3つの混合角($\theta_{12}, \theta_{13}, \theta_{23}$)と
つのCP複素位相$\delta$を使って次のように記述できる。

\begin{align}
    \pmqty{ \nu_{e} \\ \nu_{\mu} \\ \nu_{\tau} }
    =
    \pmqty{
        U_{e 1} & U_{e 2} & U_{e 3} \\
        U_{\mu 1} & U_{\mu 2} & U_{\mu 3} \\
        U_{\tau 1} & U_{\tau 2} & U_{\tau 3}
    }
    \pmqty{ \nu_{1} \\ \nu_{2} \\ \nu_{3} }
\end{align}

この$3 \times 3$行列は世代間の混合を表すユニタリ行列で、MNS(Maki-Nakagawa-Sakata)行列と呼ばれる。

\begin{align}
U_{\alpha i} =
\pmqty{
    1 & 0 & 0\\
    0 & \cos \theta_{23} & \sin \theta_{23}\\
    0 & - \sin \theta_{23} & \cos \theta_{23}
}
\pmqty{
    \cos \theta_{13} & 0 & \sin \theta_{13} e^{-i \delta}\\
    0 & 1 & 0\\
    - \sin \theta_{13} e^{i \delta} & 0 & \cos \theta_{13}
}
\pmqty{
    \cos \theta_{12} & \sin \theta_{12} & 0\\
    - \sin \theta_{12} & \cos \theta_{12} & 0\\
    0 & 0 & 1
}
\end{align}

ここで、$\alpha = (e, \mu, \tau)$、$i = (1, 2, 3)$である
\end{document}
