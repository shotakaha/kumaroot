\section{physicsパッケージ}

物理や数学で使用する記号のを簡単に入力できるパッケージです。
微分記号の\texttt{d}を自動で\texttt{mathrm}したり、
偏微分記号や行列も(だいたい)読む通りに入力できます。

\subsection{微分}

\begin{align}
    \mathrm{d}x, \mathrm{d}^{2}x, \mathrm{d}^{3}x, \cdots, \mathrm{d}^{n}x\\
    \dd{x}, \dd[2]{x}, \dd[3]{x}, \cdots, \dd[n]{x}
\end{align}

\subsection{ベクトル}

ベクトルの書き方もいろいろありますが、簡単なコマンドが用意されています。
$\vb{a}$のように太字で書いたり、
$\va{a}$のように矢印で書いたり、
$\vu{\theta}$のように単位ベクトルで書いたりできます。

内積(dot product)$\vb{a} \dotproduct \vb{b}$や
外積(cross product)$\vb{a} \crossproduct \vb{b}$もあります。

グラディエント$\grad$、
発散(ダイバージェンス)$\div$、
回転(ローテーション)$\curl$のコマンドもあります。

\begin{align}
    \div \vb{B}(t,\vb{x}) & = 0\\
    \curl \vb{E}(t,\vb{x}) & = - \pdv{\vb{B}(t,\vb{x})}{t}\\
    \div \vb{D} & = \rho(t,\vb{x})\\
    \curl \vb{H}(t,\vb{x}) & = \vb{j}(t,\vb{x}) + \pdv{\vb{D}(t,\vb{x})}{t}
\end{align}
