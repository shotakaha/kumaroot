\section{fontspecパッケージ}

\begin{minted}{latex}
    \usepackage{luatexja-fontspec}
    % 欧文フォント
    \setmainfont{KiwiMaru-Regular}
    \setsansfont{ReggaeOne-Regular}
    \setmonofont{HackGen35Console-Regular}
    % 和文フォント
    \setmainjfont{KiwiMaru-Regular}
    \setsansjfont{ReggaeOne-Regular}
    \setmonojfont{HackGen35Console-Regular}
\end{minted}

LuaLaTexではフォントを自由にカスタマイスできます。
欧文/和文フォントの両方で
\textrm{セリフ体をKiwiMaru}、
\textsf{サンセリフ体をReggaeOne}、
\texttt{モノスペース体をHackGen35Console}
にしてみました。

\texttt{luatexja-fontspec}を読み込むと、\texttt{fontspec}も同時に読み込まれます。
特に理由がない場合は、和文と欧文をそれぞれ同じに設定しておくとよいと思います。
