\section{latexmkrc}

\texttt{latexmk}はLaTeX版Makefileです。
\texttt{latexmkrc}に設定を記述しておけば、LaTeXエンジンによらず``latexmk``でコンパイルできます。

今回は\texttt{LuaLaTeX}を使いたいので
\texttt{latexmkrc}に\texttt{\$pdf\_mode = 4;}と記述しておきます。

\subsection{コンパイルするファイルを指定したい}

\texttt{latexmk ファイル名}でファイル名を指定できる。
拡張子(\texttt{.tex})はつけても、つけなくてもよい。
ファイル名を指定しない場合は、ディレクトリ内にあるTeXファイルが対象となる。

\texttt{latexmkrc}の\texttt{@default\_files}でファイルを指定することができる。

\subsection{ライブプレビューしたい}

文書に変更を加えたときに、自動でコンパイルするには
\texttt{-pvc (preview continuously)}オプションを使います。

\subsection{オプションを有効/無効にしたい}

\texttt{latexmk}のオプションには有効/無効がセットになっているものがあります。
たいていの場合\texttt{-オプション名}で有効、
\texttt{-オプション名-}で無効にできます。

たとえば\texttt{-pdf}はPDFを生成し、\texttt{-pdf-}でPDF生成をOFFにできます。

\subsection{Read the Docsのコンパイルオプション}

\begin{minted}{shell}
latexmk -r latexmkjarc
\end{minted}

\begin{description}
    \item[\texttt{-r latexmkjarc}] 設定ファイルに\texttt{latexmkjarc}を設定
    \item[\texttt{-pdfdvi}] DVIファイルからPDFを生成する
    \item[\texttt{-f}] エラーが出ても続行する
    \item[\texttt{-dvi-}] DVIを生成しない
    \item[\texttt{-ps-}] PostScriptを生成しない
    \item[\texttt{-jobname=kumaroot}] 出力ファイルのベースネームを設定
    \item[\texttt{-interaction=nonstopmode}] ユーザー入力をスキップ
\end{description}
