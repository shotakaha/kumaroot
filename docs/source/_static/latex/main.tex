\documentclass{jlreq}

\usepackage{fancyhdr}
\usepackage{fancyvrb}
\usepackage{physics}
\usepackage{siunitx}
\AtBeginDocument{\RenewCommandCopy\qty\SI}
\usepackage{hepnames}
\usepackage{hyperref}
\hypersetup{colorlinks=true}
\hypersetup{allcolors=blue}
\hypersetup{bookmarksopen=true}
\hypersetup{bookmarksnumbered=true}
\hypersetup{pdftitle=日本語でLuaLaTeX}
\hypersetup{pdfauthor=Kuma Noteの ノート}

\title{LuaLaTeX + jlreq}
\author{Quma}
\date{\today}

\begin{document}

\maketitle

\begin{abstract}
この文書はドキュメントクラスに\texttt{jlreq}を指定し、
\texttt{LuaLaTeX}を使ってビルドしています。
\end{abstract}

\tableofcontents


\section{タイプセットの設定(latexmkrc)}

\texttt{latexmk}というLaTeX版makeコマンドがあります。
タイプセットに必要な設定を\texttt{latexmkrc}に記述すると、LaTeXエンジンなどに関係なく常に\texttt{latexmk}でタイプセット(コンパイル)できるようになります。

% 読み込むファイル名は main.tex からのパスでOK
\inputminted{bash}{latexmkrc}

\subsection{エンジンを設定する(\$pdf\_mode)}

\texttt{LuaLaTeX}を使う場合は、\texttt{\$pdf\_mode = 4;}と記述します。

\subsection{コンパイルするファイルを指定したい(@default\_files)}

\texttt{@default\_files}にファイルを指定できます。
拡張子(\texttt{.tex})はつけても、つけなくてもよいです。
ファイル名を指定しない場合は、ディレクトリ内にあるTeXファイルが対象となります。

\subsection{ライブプレビューしたい(\$preview\_continuous\_mode)}

ライブプレビューをするには、\texttt{\$preview\_continuous\_mode = 1}とします。
コンパイル対象になっているファイルに変更を加えたときに、自動でタイプセットしてくれます。
コマンドラインの\texttt{-pvc (preview continuously)}オプションでも指定できます。

\subsection{シェルエスケープしたい(``\-shell\-escape``)}

タイプセット時に外部コマンドを呼び出す場合は\texttt{\-shell\-escape}オプションが必要です。
セキュリティの観点からこのオプションは設定ファイルで常時有効にするのではなく、コマンド実行時にその都度有効にするのがよいと思います。

\begin{minted}{bash}
$ latexmk -shell-escape
\end{minted}

\section{ハイパーリンク(hyperref)}

目次などにハイパーリンクを挿入するパッケージ。
読み込むパッケージの中で一番最後に記述する。

\texttt{(u)pLaTeX}を使っている場合、\texttt{pxjahyper}パッケージも追加する必要がある。
\texttt{LuaLaTeX}を使っている場合、\texttt{pxjahyper}パッケージは不要。

\subsection{しおりを作成}

目次のしおり(ブックマーク)は自動で作成されるようになっている。
LuaLaTeXを使っていると、しおりのタイトルは文字化けしない。
デフォルトだと赤枠で囲われるようになっているので、落ち着いた色に変更する。

\subsection{PDF作成者の情報}

PDF作成者の情報を追加できる。
\texttt{Preview}は\texttt{Tools -> Show Inspector}、
\texttt{Adobe Acrobat}は\texttt{File -> Properties}で確認できる。

デフォルトだと\texttt{Title}や\texttt{Author}などが空欄になっている。
それぞれ\texttt{pdftitle}と\texttt{pdfauthor}で設定できる。

\section{siunitxパッケージ}

単位や物理量を簡単に入力できるパッケージです。
SI単位系で使用する\unit{m}、\unit{kg}、\unit{s}、\unit{A}はもちろん、
それを合成した次元がマクロで定義されています。
出力結果は同じですが、数式モードを使った場合よりはるかに入力が楽ちんになります。

\texttt{physics}パッケージと一緒に使うと\texttt{qty}コマンドが重なり無効になってしまうので、対処する必要があります。

\section{physicsパッケージ}

物理や数学で使用する記号のを簡単に入力できるパッケージです。
微分記号の\texttt{d}を自動で\texttt{mathrm}したり、
偏微分記号や行列も(だいたい)読む通りに入力できます。

\subsection{微分}

\begin{align}
    \mathrm{d}x, \mathrm{d}^{2}x, \mathrm{d}^{3}x, \cdots, \mathrm{d}^{n}x\\
    \dd{x}, \dd[2]{x}, \dd[3]{x}, \cdots, \dd[n]{x}
\end{align}

\subsection{ベクトル}

ベクトルの書き方もいろいろありますが、簡単なコマンドが用意されています。
$\vb{a}$のように太字で書いたり、
$\va{a}$のように矢印で書いたり、
$\vu{\theta}$のように単位ベクトルで書いたりできます。

内積(dot product)$\vb{a} \dotproduct \vb{b}$や
外積(cross product)$\vb{a} \crossproduct \vb{b}$もあります。

グラディエント$\grad$、
発散(ダイバージェンス)$\div$、
回転(ローテーション)$\curl$のコマンドもあります。

\begin{align}
    \div \vb{B}(t,\vb{x}) & = 0\\
    \curl \vb{E}(t,\vb{x}) & = - \pdv{\vb{B}(t,\vb{x})}{t}\\
    \div \vb{D} & = \rho(t,\vb{x})\\
    \curl \vb{H}(t,\vb{x}) & = \vb{j}(t,\vb{x}) + \pdv{\vb{D}(t,\vb{x})}{t}
\end{align}

\section{hepnamesパッケージ}

素粒子名前(記号)を簡単に入力できるパッケージです。
陽電子($\Ppositron$)、
B中間子(\PB)


\end{document}
