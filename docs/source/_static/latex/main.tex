\documentclass{jlreq}
\title{LualLaTeX + jlreq}
\author{Shota}


\usepackage{physics}
\usepackage{siunitx}
\AtBeginDocument{\RenewCommandCopy\qty\SI}
\usepackage{hepnames}
\usepackage{hyperref}
\hypersetup{colorlinks=true}
\hypersetup{allcolors=blue}
\hypersetup{bookmarksopen=true}
\hypersetup{bookmarksnumbered=true}
\hypersetup{pdftitle=日本語でLuaLaTeX}
\hypersetup{pdfauthor=Kuma Noteの ノート}

\begin{document}

\tableofcontents

\section{ドキュメントクラス}

この文書はドキュメントクラスに\texttt{jlreq}を指定し、
\texttt{LuaLaTeX}を使ってビルドしています。

\section{latexmkrc}

\texttt{LuaLaTeX}を使いたいので
\texttt{latexmkrc}に\texttt{\$pdf\_mode = 4;}と書いておきます。

\section{ハイパーリンク(hyperref)}

目次などにハイパーリンクを挿入するパッケージ。
読み込むパッケージの中で一番最後に記述する。

\texttt{(u)pLaTeX}を使っている場合、\texttt{pxjahyper}パッケージも追加する必要がある。
\texttt{LuaLaTeX}を使っている場合、\texttt{pxjahyper}パッケージは不要。

\subsection{しおりを作成}

目次のしおり(ブックマーク)は自動で作成されるようになっている。
LuaLaTeXを使っていると、しおりのタイトルは文字化けしない。
デフォルトだと赤枠で囲われるようになっているので、落ち着いた色に変更する。

\subsection{PDF作成者の情報}

PDF作成者の情報を追加できる。
\texttt{Preview}は\texttt{Tools -> Show Inspector}、
\texttt{Adobe Acrobat}は\texttt{File -> Properties}で確認できる。

デフォルトだと\texttt{Title}や\texttt{Author}などが空欄になっている。
それぞれ\texttt{pdftitle}と\texttt{pdfauthor}で設定できる。

\section{siunitxパッケージ}

単位や物理量を簡単に入力できるパッケージです。
SI単位系で使用する\unit{m}、\unit{kg}、\unit{s}、\unit{A}はもちろん、
それを合成した次元がマクロで定義されています。
出力結果は同じですが、数式モードを使った場合よりはるかに入力が楽ちんになります。

\texttt{physics}パッケージと一緒に使うと\texttt{qty}コマンドが重なり無効になってしまうので、対処する必要があります。

\section{physicsパッケージ}

物理や数学で使用する記号のを簡単に入力できるパッケージです。
微分記号の\texttt{d}を自動で\texttt{mathrm}したり、
偏微分記号や行列も(だいたい)読む通りに入力できます。

\subsection{微分}

\begin{align}
    \mathrm{d}x, \mathrm{d}^{2}x, \mathrm{d}^{3}x, \cdots, \mathrm{d}^{n}x\\
    \dd{x}, \dd[2]{x}, \dd[3]{x}, \cdots, \dd[n]{x}
\end{align}

\subsection{ベクトル}

ベクトルの書き方もいろいろありますが、簡単なコマンドが用意されています。
$\vb{a}$のように太字で書いたり、
$\va{a}$のように矢印で書いたり、
$\vu{\theta}$のように単位ベクトルで書いたりできます。

内積(dot product)$\vb{a} \dotproduct \vb{b}$や
外積(cross product)$\vb{a} \crossproduct \vb{b}$もあります。

グラディエント$\grad$、
発散(ダイバージェンス)$\div$、
回転(ローテーション)$\curl$のコマンドもあります。

\begin{align}
    \div \vb{B}(t,\vb{x}) & = 0\\
    \curl \vb{E}(t,\vb{x}) & = - \pdv{\vb{B}(t,\vb{x})}{t}\\
    \div \vb{D} & = \rho(t,\vb{x})\\
    \curl \vb{H}(t,\vb{x}) & = \vb{j}(t,\vb{x}) + \pdv{\vb{D}(t,\vb{x})}{t}
\end{align}

\section{hepnamesパッケージ}

素粒子名前(記号)を簡単に入力できるパッケージです。
陽電子($\Ppositron$)、
B中間子(\PB)


\end{document}
