\documentclass{jlreq}

\usepackage{luatexja-fontspec}
\usepackage{luatexja-ruby}
% 欧文フォント
\setmainfont{KiwiMaru-Regular}
\setsansfont{ReggaeOne-Regular}
\setmonofont{HackGen35Console-Regular}
% 和文フォント
\setmainjfont{KiwiMaru-Regular}
\setsansjfont{ReggaeOne-Regular}
\setmonojfont{HackGen35Console-Regular}

\usepackage{fancyhdr}
\usepackage{fancyvrb}
\usepackage{minted}
% \usemintedstyle{dracula}
\usepackage{bxjalipsum}
\renewcommand{\thesection}{第\arabic{section}章}
\renewcommand{\thesubsection}{第\arabic{section}.\arabic{subsection}節}

\usepackage{physics}
\usepackage{siunitx}
\AtBeginDocument{\RenewCommandCopy\qty\SI}
\usepackage{hepnames}
\usepackage{hyperref}
\hypersetup{colorlinks=true}
\hypersetup{allcolors=blue}
\hypersetup{bookmarksopen=true}
\hypersetup{bookmarksnumbered=true}
\hypersetup{pdftitle=日本語でLuaLaTeX}
\hypersetup{pdfauthor=Kuma Noteの ノート}

\title{LuaLaTeX + jlreq}
\author{Quma}
\date{\today}

\begin{document}

\maketitle

\begin{abstract}
この文書はドキュメントクラスに\texttt{jlreq}を指定し、
\texttt{LuaLaTeX}を使ってビルドしています。
\end{abstract}

\tableofcontents


\section{latexmkrc}

\texttt{latexmk}はLaTeX版Makefileです。
\texttt{latexmkrc}に設定を記述しておけば、LaTeXエンジンによらず``latexmk``でコンパイルできます。

今回は\texttt{LuaLaTeX}を使いたいので
\texttt{latexmkrc}に\texttt{\$pdf\_mode = 4;}と記述しておきます。

\subsection{コンパイルするファイルを指定したい}

\texttt{latexmk ファイル名}でファイル名を指定できる。
拡張子(\texttt{.tex})はつけても、つけなくてもよい。
ファイル名を指定しない場合は、ディレクトリ内にあるTeXファイルが対象となる。

\texttt{latexmkrc}の\texttt{@default\_files}でファイルを指定することができる。

\subsection{ライブプレビューしたい}

文書に変更を加えたときに、自動でコンパイルするには
\texttt{-pvc (preview continuously)}オプションを使います。

\subsection{オプションを有効/無効にしたい}

\texttt{latexmk}のオプションには有効/無効がセットになっているものがあります。
たいていの場合\texttt{-オプション名}で有効、
\texttt{-オプション名-}で無効にできます。

たとえば\texttt{-pdf}はPDFを生成し、\texttt{-pdf-}でPDF生成をOFFにできます。

\subsection{Read the Docsのコンパイルオプション}

\begin{Verbatim}[frame=leftline]
    latexmk -r latexmkjarc
\end{Verbatim}

\begin{description}
    \item[\texttt{-r latexmkjarc}] 設定ファイルに\texttt{latexmkjarc}を設定
    \item[\texttt{-pdfdvi}] DVIファイルからPDFを生成する
    \item[\texttt{-f}] エラーが出ても続行する
    \item[\texttt{-dvi-}] DVIを生成しない
    \item[\texttt{-ps-}] PostScriptを生成しない
    \item[\texttt{-jobname=kumaroot}] 出力ファイルのベースネームを設定
    \item[\texttt{-interaction=nonstopmode}] ユーザー入力をスキップ
\end{description}

\section{fontspecパッケージ}

\begin{minted}{latex}
    \usepackage{luatexja-fontspec}
    % 欧文フォント
    \setmainfont{KiwiMaru-Regular}
    \setsansfont{ReggaeOne-Regular}
    \setmonofont{HackGen35Console-Regular}
    % 和文フォント
    \setmainjfont{KiwiMaru-Regular}
    \setsansjfont{ReggaeOne-Regular}
    \setmonojfont{HackGen35Console-Regular}
\end{minted}

LuaLaTexではフォントを自由にカスタマイスできます。
欧文/和文フォントの両方で
\textrm{セリフ体をKiwiMaru}、
\textsf{サンセリフ体をReggaeOne}、
\texttt{モノスペース体をHackGen35Console}
にしてみました。

\texttt{luatexja-fontspec}を読み込むと、\texttt{fontspec}も同時に読み込まれます。
特に理由がない場合は、和文と欧文をそれぞれ同じに設定しておくとよいと思います。

\section{fancyhdrパッケージ}

\thispagestyle{fancy}
\lhead{ヘッダー・左(チャプター名)}
\chead{ヘッダー・中央}
\rhead{ヘッダー・右(タイトル)}
\lfoot{フッター・左}
\cfoot{-\, \thepage \, -}
\rfoot{フッター・右}

文書のヘッダー/フッターをいい感じにデコレーションするパッケージです。
読み込む順番に気をつける必要があるみたいです。

現在のページにだけ適用したい場合は、\mintinline{latex}{\thispagestyle{fancy}}します。
改ページ(\mintinline{latex}{\newpage})でリセットされます。

\mintinline{latex}{\pagestyle{fancy}}を宣言すると、そのページ以降に適用されます。
すべてのページに適用する場合は、プリアンブルで宣言すればOKです。

ページのスタイルとして、全部で6ヶ所のヘッダー/フッターを設定できます。

\begin{minted}{latex}
    \thispagestyle{fancy}
    \lhead{ヘッダー・左(チャプター名)}
    \chead{ヘッダー・中央}
    \rhead{ヘッダー・右(タイトル)}
    \lfoot{フッター・左}
    \cfoot{フッター・-\, \thepage \, -}
    \rfoot{フッター・右}
\end{minted}

ここでは\mintinline{latex}{\thispagestyle{fancy}}しているので、このセクションの柱にだけヘッダー/フッターが表示されているはずです。

\section{ハイパーリンク(hyperref)}

目次などにハイパーリンクを挿入するパッケージ。
読み込むパッケージの中で一番最後に記述する。

\texttt{(u)pLaTeX}を使っている場合、\texttt{pxjahyper}パッケージも追加する必要がある。
\texttt{LuaLaTeX}を使っている場合、\texttt{pxjahyper}パッケージは不要。

\subsection{しおりを作成}

目次のしおり(ブックマーク)は自動で作成されるようになっている。
LuaLaTeXを使っていると、しおりのタイトルは文字化けしない。
デフォルトだと赤枠で囲われるようになっているので、落ち着いた色に変更する。

\subsection{PDF作成者の情報}

PDF作成者の情報を追加できる。
\texttt{Preview}は\texttt{Tools -> Show Inspector}、
\texttt{Adobe Acrobat}は\texttt{File -> Properties}で確認できる。

デフォルトだと\texttt{Title}や\texttt{Author}などが空欄になっている。
それぞれ\texttt{pdftitle}と\texttt{pdfauthor}で設定できる。

\section{siunitxパッケージ}

単位や物理量を簡単に入力できるパッケージです。
SI単位系で使用する\unit{m}、\unit{kg}、\unit{s}、\unit{A}はもちろん、
それを合成した次元がマクロで定義されています。
出力結果は同じですが、数式モードを使った場合よりはるかに入力が楽ちんになります。

\texttt{physics}パッケージと一緒に使うと\texttt{qty}コマンドが重なり無効になってしまうので、対処する必要があります。

\section{physicsパッケージ}

物理や数学で使用する記号のを簡単に入力できるパッケージです。
微分記号の\texttt{d}を自動で\texttt{mathrm}したり、
偏微分記号や行列も(だいたい)読む通りに入力できます。

\subsection{微分}

\begin{align}
    \mathrm{d}x, \mathrm{d}^{2}x, \mathrm{d}^{3}x, \cdots, \mathrm{d}^{n}x\\
    \dd{x}, \dd[2]{x}, \dd[3]{x}, \cdots, \dd[n]{x}
\end{align}

\subsection{ベクトル}

ベクトルの書き方もいろいろありますが、簡単なコマンドが用意されています。
$\vb{a}$のように太字で書いたり、
$\va{a}$のように矢印で書いたり、
$\vu{\theta}$のように単位ベクトルで書いたりできます。

内積(dot product)$\vb{a} \dotproduct \vb{b}$や
外積(cross product)$\vb{a} \crossproduct \vb{b}$もあります。

グラディエント$\grad$、
発散(ダイバージェンス)$\div$、
回転(ローテーション)$\curl$のコマンドもあります。

\begin{align}
    \div \vb{B}(t,\vb{x}) & = 0\\
    \curl \vb{E}(t,\vb{x}) & = - \pdv{\vb{B}(t,\vb{x})}{t}\\
    \div \vb{D} & = \rho(t,\vb{x})\\
    \curl \vb{H}(t,\vb{x}) & = \vb{j}(t,\vb{x}) + \pdv{\vb{D}(t,\vb{x})}{t}
\end{align}

\section{hepnamesパッケージ}

素粒子名前(記号)を簡単に入力できるパッケージです。
陽電子($\Ppositron$)、
B中間子(\PB)


\section{minted}

文書内にコードサンプルを表示したいときの環境です。
コンパイルする際には\texttt{-shell-escape}オプションが必要です。
(\mintinline{shell}{$ latexmk -pvc -shell-escape})

\begin{listing}[H]
\begin{minted}[linenos=true,showspaces]{python}
from dataclasses import dataclass
from pathlib import Path

@dataclass
class Config:
    path

    def __post_init__(self):
        self.path = Path(self.path)

if __name__ == "__main__":
    c = Config("config.toml")

\end{minted}
\caption{Pythonのサンプル}
\end{listing}

\begin{listing}[H]
\begin{minted}{latex}
\documentclass{jlreq}
\usepackage{minted}
\begin{document}

\end{document}
\end{minted}
\caption{LaTeXのサンプル}
\end{listing}

\section{luatexja-rubyパッケージ}

\ruby{LaTeX}{らてふ}でルビを\ruby{振}{ふ}る\ruby{場|合}{ば|あい}は\texttt{luatexja-ruby}パッケージを\ruby{使}{つか}います。

\begin{listing}
\begin{minted}{latex}
    soredemo
\end{minted}
\caption{ルビのサンプル}
\end{listing}




\ltjruby{裸足}{はだし}になって どうするつもり?
そのまま どこかへ\ltjruby{歩}{ある}いて\ltjruby{行}{い}くの?
ねえ \ltjruby{何}{なに}をしたいんだ?
\ltjruby{行|動}{こう|どう}が\ltjruby{予|測}{よ|そく}できないよ
\ltjruby{他人}{ひと}の\ltjruby{目}{め} \ltjruby{気}{き}にせずに\ltjruby{気}{き}まぐれで...
そう\ltjruby{君}{きみ}にいつも \ltjruby{振}{ふ}り\ltjruby{回}{まわ}されて
あきれたり \ltjruby{疲}{つか}れたり
それでも\ltjruby{君}{きみ}に\ltjruby{恋}{こい}してる

\section{bxjalipsum パッケージ}

\subsection{いろはにほへと}

\jalipsum{iroha}

\subsection{じゅげむ}

\jalipsum{jugemu}

\jalipsum{jugemuP}

\subsection{吾輩は猫である}

\jalipsum[1-3]{wagahai}

\subsection{初恋}

\jalipsum[1-3]{hatsukoi}

\subsection{草枕}

\jalipsum[1-3]{kusamakura}


\end{document}
